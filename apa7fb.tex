% Se redefinen comandos de flexbib para adaptarse a los gustos de APA7

\renewcommand{\bibnamefont}[1]{#1}    % fuente normal en apellidos
\renewcommand{\bibfnamefont}[1]{#1}   % fuente normal en nombre de pila. Usar iniciales  en BBDD
\renewcommand{\enquote}[1]{#1}        % elimina comillas del título
\renewcommand{\bibvolfont}[1]{\textit{#1}}     % volumen en itálica
\def\bibnumfont#1{\upshape{#1}}       % número en fuente normal
\renewcommand{\nameseparator}{,}      % separa los autores con , 
\renewcommand{\bbland}{\&}            % el último de forma especial
\renewcommand{\namereplace}[1]{#1}    % con autores repetidos, los pone en lugar de una raya
\renewcommand{\bblpp}{}               % elimina pp. de las páginas en article, no en inproceedings
\renewcommand{\bblp}{}                % elimina p. de las páginas en article, no en inproceedings
\let\cite\citealp                     % cambia formato de cita para adaptarse al de APA7 
                 % APA7 original parece ignorar el campo    address
                 % en  inbook  tengo experiencias de que ignora los campos chapter y pages ???
                 % Como flexbib respeta el nombre de pila siempre y otros ponen iniciales,
                 % para integrarse simular a otros desde flexbib una opción es repetir registros
                 % uno a tu gusto (con un  alias)  y otro con alias-I que coincide con el anterior
                 % salvo que para el nombre de pila se usan iniciales  (-I)
